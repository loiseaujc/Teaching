%\documentclass[12pt, answers]{exam}
\documentclass[12pt]{exam}
\usepackage[utf8]{inputenc}

\usepackage[margin=1in]{geometry}
\usepackage{amsmath,amssymb}
\usepackage{multicol}
\usepackage[]{graphicx}
\usepackage[]{bm}
\usepackage[]{nicefrac}
\usepackage[]{xcolor}

\newcommand{\class}{Physique non-linéaire}
\newcommand{\term}{Master 2}
\newcommand{\examnum}{Examen final}
\newcommand{\examdate}{Février 2022}
\newcommand{\timelimit}{120 Minutes}

\pagestyle{head}
\firstpageheader{}{}{}
\runningheader{\class}{\examnum\ - Page \thepage\ of \numpages}{\examdate}
\runningheadrule

\usepackage[]{tikz}

\begin{document}

\noindent
\begin{tabular*}{\textwidth}{l @{\extracolsep{\fill}} r @{\extracolsep{6pt}} l}
% \textbf{\class} & \textbf{Nom -- Prénom:} & \makebox[2in]{\hrulefill}\\
% \textbf{\term} &&\\
\textbf{\examnum} &&\\
\textbf{\examdate} &&
% \textbf{Time Limit: \timelimit} & Teaching Assistant & \makebox[2in]{\hrulefill}
\textbf{Durée: \timelimit} %&&
\end{tabular*}\\
\rule[2ex]{\textwidth}{2pt}

Ce sujet contient \numpages\ pages (en comptant la page de garde) et \numquestions\ exercices.\\
Le nombre total de point est de \numpoints.

\begin{center}
  Barême\\
  \bigskip
  \addpoints
  \gradetable[v][questions]
\end{center}

\noindent
\rule[2ex]{\textwidth}{2pt}

\begin{questions}

  \question[10] \textbf{Exercice n°1}
  \noaddpoints

  Pour chacun des exemples ci-dessous, tracez qualitativement la ligne de phase pour différentes valeurs du paramètre de contrôle $\mu$.
  Déterminez pour quelle valeur de $\mu$ une bifurcation a lieu.
  Tracez ensuite le diagramme de bifurcation et déterminez de quel type de bifurcation il s'agit.

  \begin{multicols}{3}
    \begin{itemize}
    \item $\dot{x} = \mu - 3x^2$
    \item $\dot{x} = 5 - \mu e^{-x^2}$
    \item $\dot{x} = \mu x - \dfrac{x}{1+x}$
    \item $\dot{x} = x + \tanh(\mu x)$
    \item $\dot{x} = \mu x + \dfrac{x}{1+x^2}$
    \item $\dot{x} = \mu x + \dfrac{x^3}{1 + x^2}$
    \end{itemize}
  \end{multicols}

  \addpoints
  \question[10] \textbf{Exercice n°2}
  \noaddpoints

  Considérons le système mécanique formé d'une barre rigide placée verticalement avec un ressort de torsion à sa base appliquant une force proportionelle à l'angle que fait le barre avec la verticale.
  Un amortisseur est également présent dont la force de frottement est proportionnelle à la vitesse angulaire.
  Enfin, une charge est appliquée verticallement au sommet de la barre.
  En partant des principes de Newton, l'équation gouvernant la dynamique du système est donnée par
  %
  \[
  I \dfrac{d^2\theta}{dt^2} + \beta \dfrac{d\theta}{dt} + k \theta - PL \sin(\theta) = 0
  \]
  %
  avec $I$ le moment d'inertie du système, $k$ la constante de raideur du ressort, $P$ le poids exercée par la masse, $L$ la longueur de la barre et $\beta$ le coefficient d'amortissement.
  L'objectif de cet exercice est de déterminer le type de bifurcations rencontrées par le système lorsque l'on fait varier la charge appliquée.

  \begin{parts}
    \part[2] Montrez que les équations du mouvement sont équivariantes vis à vis de la transformation $\theta \mapsto -\theta$.
    Quelle conséquence cette équivariance a-t'elle vis à vis des propriétés du système ?

    \part[4] En posant $\mu = \nicefrac{PL}{k}$, les états d'équilibre du système sont solutions de l'équation suivante
    %
    \[
    \theta = \mu \sin(\theta).
    \]
    %
    Déterminer le nombre de point(s) fixe(s) du système en fonction de $\mu$, étudiez leur stabilité linéaire et tracer le diagramme de bifurcation correspondant.
    De quel type de bifurcation s'agit-il ?

    \part[4] Comparé à cette situation idéalisée, un système réel présente toujours de petites imperfections qui brisent la symétrie.
    Refaite l'analyse de la question précédente pour le système suivant
    %
    \[
    I \dfrac{d^2 \theta}{dt^2} + \beta \dfrac{d\theta}{dt} + k \theta - PL \left( h + \sin(\theta) \right) = 0
    \]
    %
    où $h$ modélise une petite imperfection (e.g. une distribution non-uniforme de la masse appliquée).
    Comment évolue votre diagramme de bifurcation en fonction de $h$ ?
    De quel phénomène s'agit-il ?

  \end{parts}

  \addpoints
  \question[10] \textbf{Exercice n°3}
  \noaddpoints

  La dynamique du lâcher tourbillonnaire derrière un cylindre bi-dimensionel peut être modélisé à l'aide des équations suivantes
  %
  \[
  \begin{aligned}
    \dot{x} & = \sigma x - y - xz \\
    \dot{y} & = x + \sigma y - yz \\
    \dot{z} & = -z + x^2 + y^2
  \end{aligned}
  \]
  %
  où $x(t)$ et $y(t)$ décrivent l'amplitude des parties réelles et imaginaires du mode instable et $z(t)$ caractérise la distortion du champ porteur, i.e. la différence entre le champ de base et le champ moyen.
  L'objectif de cet exercice est de dériver des équations d'amplitude pour le lâcher tourbillonnaire.

  \begin{parts}
    \part[1] En introduisant la variable complexe $\eta = x + i y$, montrez dans un premier temps que les équations peuvent se mettre sous la forme
    %
    \[
    \begin{aligned}
      \dot{\eta} & = \left( \sigma + i \right) \eta - \eta z \\
      \dot{z} & = -z + \vert \eta \vert^2.
    \end{aligned}
    \]
    %
    Ensuite, en posant $\eta(t) = r(t) \exp\left( i \varphi(t) \right)$ où $r$ est l'amplitude du lâcher tourbillonnaire et $\varphi$ sa phase, montrez que cela implique
    %
    \[
    \begin{aligned}
      \dot{r} & = \sigma r - rz \\
      \dot{z} & = -z + r^2
    \end{aligned}
    \]
    %
    tandis que la phase est simplement donnée par $\varphi(t) = \varphi_0 + t$.
    Par analogie avec les équations de Navier-Stokes, quel rôle joue le terme $r^2$ dans la seconde équation ?

    \part[4] En posant $\sigma = \varepsilon^2$ et en introduisant l'échelle de temps super-lente $\tau = \epsilon^2 t$, montrez qu'aux différents ordres, on obtient les équations suivantes
    %
    \[
    \begin{aligned}
      \mathcal{O}(\varepsilon) : & \quad \dfrac{\partial r_1}{\partial t} = 0 \\
      & \quad \dfrac{\partial z_1}{\partial t} = -z_1\\
      %
      \mathcal{O}(\varepsilon^2) : & \quad \dfrac{\partial r_2}{\partial t} = -r_1 z_1 \\
      & \quad \dfrac{\partial z_2}{\partial t} = - z_2 + r_1^2 \\
      %
      \mathcal{O}(\varepsilon^3) : & \quad \dfrac{\partial r_3}{\partial t} = r_1 + r_1 z_2 + r_2 z_1 - \dfrac{\partial r_1}{\partial \tau} \\
      & \quad \dfrac{\partial z_3}{\partial t} = - z_3 + 2 r_1 r_2 - \dfrac{\partial z_1}{\partial \tau}
    \end{aligned}
    \]
    %
    Notez qu'à l'ordre $0$ la solution est donnée par $(r_0, z_0) = (0, 0)$ (notre point fixe).
    Ainsi, votre développement asymptotique doit être de la forme $q(t, \tau) = \epsilon q_1(t, \tau) + \epsilon^2 q_2(t, \tau) + \epsilon^3 q_3(t, \tau) + \cdots$.

    \part[2] La solution à l'ordre 1 est donnée par $r_1(t, \tau) = A(\tau)$.
    Montrez qu'à l'ordre 2 on a $z_2(t, \tau) = A^2(\tau) \left(1 - e^{-t} \right)$.

    \part[2] En vous basant sur ces résultats, justifiez que l'amplitude du lâcher tourbillonnaire est gouvernée par
    %
    \[
    \dfrac{d A}{d\tau} = A - A^3.
    \]
    %
    Tracez à la main à quoi ressemble l'évolution de $A(t)$.

    \part[1] Par analogie avec les équations de Navier-Stokes, expliquez par quel mécanisme l'amplitude $A(\tau)$ du lâcher tourbillonnaire finit-elle par saturer.

  \end{parts}

\end{questions}

\end{document}
