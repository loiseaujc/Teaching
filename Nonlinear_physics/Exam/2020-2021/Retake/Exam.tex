\documentclass[12pt, answers]{exam}
\usepackage[utf8]{inputenc}

\usepackage[margin=1in]{geometry}
\usepackage{amsmath,amssymb}
\usepackage{multicol}
\usepackage[]{graphicx}
\usepackage[]{bm}
\usepackage[]{nicefrac}
\usepackage[]{xcolor}

\newcommand{\class}{Physique non-linéaire}
\newcommand{\term}{Master 2}
\newcommand{\examnum}{Examen final}
\newcommand{\examdate}{Février 2021}
\newcommand{\timelimit}{120 Minutes}

\pagestyle{head}
\firstpageheader{}{}{}
\runningheader{\class}{\examnum\ - Page \thepage\ of \numpages}{\examdate}
\runningheadrule

\usepackage[]{tikz}

\begin{document}

\noindent
\begin{tabular*}{\textwidth}{l @{\extracolsep{\fill}} r @{\extracolsep{6pt}} l}
% \textbf{\class} & \textbf{Nom -- Prénom:} & \makebox[2in]{\hrulefill}\\
% \textbf{\term} &&\\
\textbf{\examnum} &&\\
\textbf{\examdate} &&
% \textbf{Time Limit: \timelimit} & Teaching Assistant & \makebox[2in]{\hrulefill}
\textbf{Durée: \timelimit} %&&
\end{tabular*}\\
\rule[2ex]{\textwidth}{2pt}

Ce sujet contient \numpages\ pages (en comptant la page de garde) et \numquestions\ exercices.\\
Le nombre total de point est de \numpoints.

\begin{center}
  Barême\\
  \bigskip
  \addpoints
  \gradetable[v][questions]
\end{center}

\noindent
\rule[2ex]{\textwidth}{2pt}

\begin{questions}

  \question[10] \textbf{Exercice 1}
  \noaddpoints

  Considérons un pendule simple de masse $m$ et de longueur $L$ dans le champ gravitationel ($g$) de la Terre and forcé par un couple constant $\Gamma$.
  En partant des principes de Newton, l'équation de la dynamique s'écrit
  %
  \[
  m L^2 \ddot{\theta} + b \dot{\theta} + mgL \sin(\theta) = \Gamma.
  \]
  %
  L'objectif est alors de déterminer le type de bifurcation rencontrée par le système en fonction du couple $\Gamma$ appliqué.

  \begin{parts}
    \part En introduisant $\tau$ telle que $t \mapsto \tau t$, montrez qu'il y'a deux choix possibles pour définir cette échelle de temps et discutez de leurs implications physiques.

    \part En supposant que l'on se trouve dans la situation où les forces de frottement dominent, l'équation de la dynamique peut se réduire (au moins en première approximation) à
    %
    \[
    \dot{\theta} = \gamma - \sin(\theta)
    \]
    %
    avec $\gamma = \nicefrac{\Gamma}{mgL}$.
    Faites un schéma de la ligne de phase du système pour différentes valeurs de $\gamma$.

    \part Lorsque $\gamma \to 1_{+}$, un point fixe métastable est créé en $\theta = \nicefrac{\pi}{2}$.
    Déterminez le type de bifurcation rencontrée.

    \part Pour $\vert \gamma \vert < 1$, le système possède deux états d'équilibre.
    Discutez de leur stabilité et de leur interprétation physique.
  \end{parts}

\end{questions}

\end{document}
