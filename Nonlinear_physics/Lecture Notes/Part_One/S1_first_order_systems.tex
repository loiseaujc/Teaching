\chapter{First-order systems}\label{chap: first-order systems}

\section{Dynamical systems on the real number line}\label{sec: dyn sys on the real number line}

\subsection{List of problems}

\begin{problem}
  Consider the simple resistor-capacitor circuit shown in figure~\ref{fig: RC circuit}.
  The equation governing the evolution of the voltage $Q(t)$ accross the capacitor is given by
  %
  \[
  \dot{Q} = - \dfrac{Q}{RC} + \dfrac{V_0}{RC}
  \]
  %
  where $R$ is the resistance, $C$ is the capacitance and $V_0 > 0$ the voltage of the power supply.
  For $t < 0$, the voltage accross the capacitor is assumed to be $0$.
  At $t = 0$, the switch is closed and current starts to flow in the circuit.

  \bigskip

  \begin{enumerate}
  \item[a)] Show graphically that the system has a single fixed point.
  \item[b)] Show that the equation above can be recast as
    %
    \[
    \dfrac{dx}{d\tau} = 1 - x
    \]
    %
    if one introduces the rescaled variable $Q(t) = V_0 x(t)$ and time $t = RC \tau$.
  \item[c)] Solve the equation analytically and show that
    %
    \[
    \lim_{\tau \to +\infty} x(\tau) = 1
    \]
    %
    and give the expression for $Q(t)$.
  \end{enumerate}
\end{problem}

\begin{figure}
  \centering
  \begin{circuitikz}
    \draw (0, 0) to[battery1, invert] (0, 3) to [nos] (2.5, 3);
    \draw (2.5, 3) to [R, i>^=$i(t)$, l=$R$] (2.5, 0);
    \draw (2.5, 0) to [C, l=$C$] (0, 0);
  \end{circuitikz}
  \caption{RC circuit for Problem 1.1}\label{fig: RC circuit}
\end{figure}

\bigskip

\begin{problem}
  Consider the following population growth model
  %
  \[
  \dot{P} = \mu P \left( 1 - \dfrac{P}{N} \right)
  \]
  %
  known as the logistic equation (also sometimes called the Verhulst model after Pierre François Verhulst who proposed it during the 1840's).
  It can be used to model the growth of a population in an environment with limited food supply such as fish in a small lake.
  In that context, $P(t)$ denotes the current population size, $\mu > 0$ its reproduction rate and $N$ is the \emph{carrying capacity}, i.e.\ the maximum population size that can be sustained in that specific environment.

  \bigskip

  \begin{enumerate}
  \item[a)] Introducing the rescaled population size $x(t)$ such that $P(t) = N x(t)$, show that the governing equation for $x(t)$ is
    %
    \[
    \dot{x} = \mu x \left( 1 - x \right).
    \]

  \item[b)] Sketch the phase line of the system assuming $\mu > 0$.
    How does the population evolves if $x(0)$ is only slightly larger than 0 ?
    Same question for $x(0)$ much larger than unity.

  \item[c)] This logistic model belongs to the class of \emph{Bernoulli equations} which can be solved analytically.
    Introducing the change of variable $y = \nicefrac{1}{x}$, show that the equation for $y(t)$ is linear.

  \item[d)] Solve the equation for $y(t)$ and thus for $x(t)$.
    Are the predictions of your solution for $x(0) \simeq 0$ and $x(0) \gg 1$ consistent with your intuition gained from the phase line ?
  \end{enumerate}
\end{problem}

\section{Elementary bifurcations}\label{sec: elementary bifurcations}

\subsection{List of problems}

\begin{problem}
  Consider the following system
  %
  \[
  \dot{x} = -x \left( x^2 - 2x - \mu \right)
  \]
  %
  where $x \in \mathbb{R}$ is the state of the system and $\mu \in \mathbb{R}$ is our control parameter.

  \bigskip

  \begin{enumerate}
  \item[a)] Compute all the branches of solutions and discuss their existence and stability properties.
  \item[b)] Show that a saddle-node bifurcation occurs at $\mu = -1$ and a transcritical one at $\mu = 0$.
  \item[c)] Sketch the bifurcation diagram of the system.
  \end{enumerate}
\end{problem}

\bigskip

\begin{problem}
  Consider a simple pendulum of mass $m$ and length $L$ in the gravitational field of earth ($g$) and driven by a constant torque $\Gamma$.
  Starting from Newton's principles, the equation of motion reads
  %
  \[
  m L^2 \ddot{\theta} + b \dot{\theta} + mgL \sin(\theta) = \Gamma.
  \]
  %
  Although it is a second-order equation (i.e.\ it involves a second derivative with respect to time), we'll see that under certain assumptions about the system it can be approximated by a first-order equation.
  Our aim will be to determine the type of bifurcation experienced by the system as the applied torque $\Gamma$ is varied.

  \bigskip

  \begin{enumerate}
  \item[a)] Introducing a time scale $\tau$ such that $t \mapsto \tau t$, show that we have the following choices for $\tau$ :
    %
    \[
    \tau = \sqrt{\dfrac{L}{g}} \quad \text{and} \quad \tau = \dfrac{b}{mgL}
    \]
    %
    and discuss their physical meaning.

  \item[b)] If we are in the over-damped situation, the equation of motion can be reduced to
    %
    \[
    \dot{\theta}  = \gamma - \sin(\theta)
    \]
    %
    where $\gamma = \nicefrac{\Gamma}{mgL}$.
    Sketch the phase line of the system for different values of the control parameter $\gamma$.

  \item[c)] As $\gamma \to 1$ from above, a meta-stable fixed point is created at $\theta = \nicefrac{\pi}{2}$.
    Introducing $\mu = \gamma - 1$ and using a second-order Taylor expansion of the equation in the vicinity of this fixed point, show that the dynamics of a small perturbation can be approximated by
    %
    \[
    \dot{\eta} = \mu + \dfrac{1}{2}\eta^2.
    \]
    %
    What type of bifurcation is this ?

  \item[d)] For $\vert \gamma \vert < 1$, the system admits two equilibrium solutions.
    Discuss their stability properties and their physical interpretation.
  \end{enumerate}
\end{problem}

\bigskip

\begin{problem}
  Consider the following mechanical system : an inverted pendulum made of a massless rigid rod is pivoted about its lower end with a torsion spring providing a restoring torque proportional to the angular displacement from the vertical equilibrium.
  A load, in the form of an attached mass, is applied vertically at the top of the rod.
  Starting from Newton's principle, the equation of motion is
  %
  \[
  I \ddot{\theta} + \beta \dot{\theta} + k \theta - PL \sin(\theta) = 0
  \]
  %
  where $I$ the moment of inertia of the system, $k$ is the torsional spring constant, $P$ is the applied load, $L$ is the length of the rod and $\beta$ is the damping coefficient.
  Our goal will be to determine when the system bifurcates and what type of bifurcation it experiences as the applied load $P$ is varied.

  \bigskip

  \begin{enumerate}

  \item[a)] Show that the equation of motion is invariant with respect to the transformation $\theta \mapsto -\theta$.
    What does this invariance tells you about the properties of the system ?

  \item[b)] As before, we'll consider the over-damped situation (i.e. friction dominates).
    Once time has been rescaled using $\tau = \nicefrac{\beta}{PL}$, the equation of motion reduces to
    %
    \[
    \dot{\theta} + \dfrac{k}{PL}\theta - \sin(\theta) = 0.
    \]
    %
    Using a third-order Taylor expansion of $\sin(\theta)$ around $\theta = 0$, show that the system experiences a bifurcation for $\nicefrac{k}{PL} = 1$.
    Sketch the bifurcation diagram.
    What type of bifurcation is this ?

  \item[c)] In a realistic system, one may have some imperfections breaking the symmetry.
    Redo the same analysis for the perturbed system
    %
    \[
    I \ddot{\theta} + \beta \dot{\theta} + k \theta - PL \left( h + \sin(\theta) \right) = 0
    \]
    %
    where $h$ models a small imbalance in the applied load.
    You can assume that $h > 0$.
    How does the bifurcation change ?
  \end{enumerate}

\end{problem}

\bigskip

\begin{problem}
  Let consider once more the evolution of a fish population model by the following logistic growth equation
  %
  \[
  \dot{P} = r P \left( 1 - \dfrac{P}{N} \right)
  \]
  %
  where $P(t) \in \mathbb{R}$ is the fish population, $r$ is the growth rate due to reproduction and $N$ is the carrying capacity of the environment.
  In this exercise, we'll investigate the influence of different harvesting strategies on the long time evolution of the fish population.
  Two different strategies will be considered.

  \paragraph{Constant harvesting}
  The first strategy considered is constant harvesting.
  In this strategy, fish are harvested at a constant rate irrespective of the current population level.
  If we denote by $H$ this rate, our model becomes
  %
  \[
  \dot{P} = r P \left( 1 - \dfrac{P}{N} \right) - H
  \]
  %
  supplemented with the initial condition $P(0) = P_0$.

  \bigskip

  \begin{enumerate}
  \item[a)] The model above has three parameters : the reproduction rate $r$, the carrying capacity $N$ and the harvesting rate $H$.
    Using a suitable rescaling, show that one of the parameter can be eliminated so that our model can be recast as
    %
    \[
    \dot{x} = r x \left( 1 - x \right) - h.
    \]

  \item[b)] Assuming $r$ is a positive constant, compute the fixed points of the model and study their linear stability properties.

  \item[c)] Sketch the bifurcation diagram.
    What type of bifurcation is this ?
    From a biological point of view, what happens to the fish population when $h$ is too large ?

  \end{enumerate}

  \paragraph{Proportional harvesting}
  In this second strategy, fish are harvested at a rate proportional to the current population level.
  Our population model then becomes
  %
  \[
  \dot{P} = r P \left( 1 - \dfrac{P}{N} \right) - HP.
  \]

  \begin{enumerate}
  \item[a)] As for the previous model, show that the current one can be written as
    %
    \[
    \dot{x} = r x \left( 1 - x \right) - hx
    \]
    %
    using a suitable rescaling.

  \item[b)] Assuming $r$ is a positive constant, compute the fixed points of the model and study their linear stability properties.

  \item[c)] Sketch the bifurcation diagram.
    What type of bifurcation is this ?
    From a biological point of view, what happens to the fish population when $h$ is too large ?
    How is this different from the previous strategy ?

  \end{enumerate}
\end{problem}
