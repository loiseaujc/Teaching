\section{Oscillators}

\subsection{The Poincaré-Lindstedt method}

\subsection{Weakly nonlinear self-sustained oscillators}

\subsection{Strongly nonlinear oscillators and relaxation oscillations}

\subsection{List of problems}

\subsubsection*{Poincaré-Lindstedt method}

\begin{problem}
  Consider the conservative Duffing oscillator whose governing equation is given by
  %
  \[
  \ddot{x} + x + \epsilon x^3 = 0 \quad \text{with initial conditions} \quad x(0) = 1, \quad \dot{x}(0) = 0
  \]
  %
  where $\epsilon$ is a small real number.
  The properties of this sytem and its fixed points have already been studied qualitatively in the previous section.
  We now aim for a more quantitative analysis.

  \bigskip

  \begin{enumerate}
  \item[a)] Introducing a strained time variable $\tau = \omega t$ with $\omega = \left( 1 + \epsilon \omega_1 + \cdots\right)$ and assuming that $x(\tau) = x_0(\tau) + \epsilon x_1(\tau) + \cdots$, compute the zeroth and first order approximation of the periodic solution.

  \item[b)] Sketch the evolution of the frequencies in the system as a function of $\epsilon$ (consider both positive and negative values of $\epsilon$).

  \item[c)] Write down a computer program that simulate the system above for different values of $\epsilon$.
    For each value, compare the computer-generated trajectories in phase plane with your Poincaré-Lindstedt approximation.
    Compare the frequencies extracted from your simulation using the Fourier transform with the results from your theoretical analysis.
  \end{enumerate}
  
\end{problem}

\bigskip

\begin{problem}
  When discussing the Poincaré-Lindstedt method for the van der Pol oscillator
  %
  \[
  \ddot{x} + \epsilon \left( x^2 - 1 \right) \dot{x} + x = 0
  \]
  %
  we have seen that, when $\epsilon$ is sufficiently small, the oscillation frequency was given by $\omega = 1 + O(\epsilon^2)$.
  Assuming now that
  %
  \[
  \omega = 1 + \epsilon \omega_1 + \epsilon^2 \omega_2 + O(\epsilon^3)
  \]
  %
  and
  %
  \[
  x(t) = x_0(t) + \epsilon x_1(t) + \epsilon^2 x_2(t) + O(\epsilon^3),
  \]
  %
  show that the frequency is actually given by $\omega = 1 - \dfrac{1}{16} \epsilon^2 + O(\epsilon^3)$.
\end{problem}

\subsubsection*{Weakly nonlinear oscillators}

\begin{problem}[A simple model for the Bénard-von Kàrmàn vortex street]
  Consider the following third-order system
  %
  \[
  \begin{aligned}
    \dot{x} & = \sigma x - y - xz - \alpha yz\\
    \dot{y} & = x + \sigma y - yz + \alpha xz\\
    \dot{z} & = -z + x^2 + y^2.
  \end{aligned}
  \]
  %
  This model arises as a simplified representation of the Bénard-von K\`arm\`an vortex street in the wake of a two-dimensional cylinder at low Reynolds numbers.
  Here, $x$ and $y$ represent the two-degrees of freedom needed to describe the oscillatory behaviour of the flow while $z$ is known as the \emph{shift-mode amplitude}.
  It describes how different the linearly unstable base flow is from the time-averaged mean flow.
  The term $x^2 + y^2$ in the equation for $z$ models the influence of the fluctuation's Reynolds stresses on the mean flow.

  \begin{enumerate}
  \item[a)] Show that this dynamical system is equivariant with respect to the transformation
    %
    \[
    \gamma
    =
    \begin{bmatrix}
      \cos(\theta) & \sin(\theta) & 0 \\
      -\sin(\theta) & \cos(\theta) & 0 \\
      0 & 0 & 1
    \end{bmatrix}
    \]
    %
    where $\gamma$ represents a rotation around the $z$-axis.

  \item[b)] Introducing the complex variable $\eta = x + i y = r e^{i\phi}$, where $r$ is the radius of oscillation and $\phi$ the phase, express the system in the cylindrical coordinates system $(r, \phi, z)$.

  \item[c)] Compute the fixed points in the $(r, z)$ plane.
    You should obtain $(r_0, z_0) = (0, 0)$ and $(r_1, z_1) = (\sqrt{\sigma}, \sigma)$.
    From a physical point of view, to what state does each of these fixed points correspond to ?

  \item[d)] Show that the origin $(r_0, z_0)$ is linearly unstable while the mean flow $(r_1, z_1)$ is linearly stable.

  \item[e)] The equation for the phase reads
    %
    \[
    \dot{\phi} = 1 + \alpha z.
    \]
    %
    How does the oscillation frequency $\omega(t) = 1 + \alpha z(t)$ evolves as the system transitions from its unstable equilibrium towards its limit cycle ?

  \item[f)] Sketch a three-dimensional phase portrait in the coordinate system $(x, y, z)$.
  \end{enumerate}
\end{problem}

\bigskip

\begin{problem}[The Bénard-von Kàrmàn vortex street revisited]
  Let us revisit the simple model for the Bénard-von Kàrmàn vortex street.
  Our goal will now be to derive an amplitude equation for the oscillation and the phase using a weakly nonlinear expansion in the vicinity of the fixed point $(x, y, z) = (0, 0, 0)$.

  \begin{enumerate}
  \item[a)] Introducing the complex variable $\eta = x + i y$, write down the model for $\eta$ and $z$.

  \item[b)] Let us introduce a slow time scale $\tau = \epsilon^2 t$ with $\epsilon^2 = \sigma$.
    Expanding the solution $(\eta, z)$ in the vicinity of $(\eta_0, z_0) = (0, 0)$ as
    %
    \[
    \begin{aligned}
      \eta(t, \tau) & = \epsilon \eta_1(t, \tau) + \epsilon^2 \eta_2(t, \tau) + \epsilon^3 \eta_3(t, \tau) + O(\epsilon^4)\\
      z(t, \tau) & = \epsilon z_1(t, \tau) + \epsilon^2 z_2(t, \tau) + \epsilon^3 z_3(t, \tau) + O(\epsilon^4),
    \end{aligned}
    \]
    %
    show that you obtain the following set of equations for each order
    %
    \[
    \begin{aligned}
      O(\epsilon) : \quad & \dfrac{d\eta_1}{dt} = i \eta_1 \\
      & \dfrac{dz_1}{dt} = - z_1 \\
      O(\epsilon^2) : \quad & \dfrac{d\eta_2}{dt} = i \eta_2 - (1 - i\alpha) \eta_1 z_1 \\
      & \dfrac{dz_2}{dt} = - z_2 + \vert \eta_1 \vert^2 \\
      O(\epsilon^3) : \quad & \dfrac{d\eta_3}{dt} = i \eta_3 + \eta_1 - (1 - i\alpha)(\eta_1 z_2 + \eta_2 z_1) - \dfrac{d\eta_1}{d\tau} \\
      & \dfrac{dz_3}{dt} = - z_3 + \overline{\eta}_1 \eta_2 + \eta_1 \overline{\eta}_2 - \dfrac{dz_1}{d\tau}.
    \end{aligned}
    \]

  \item[c)] The solution for $\eta_1$ is \( \eta_1(t, \tau) = A(\tau) e^{it}\).
    Show that \( z_2(t, \tau) = \vert A(\tau) \vert^2 \left( 1 - e^{-t} \right) \) (assuming $z_2(0, 0) = 0$).

  \item[d)] Introducing these two expressions in the equation for $\eta_3$, show that invoking the Fredholm theorem to kill the resonant terms leads to
    %
    \[
    \dfrac{dA}{d\tau} = A - \left( 1 - i\alpha \right) \vert A \vert^2 A.
    \]
    %
    Expressing the complex amplitude as $A(\tau) = R(\tau) e^{i \phi(\tau)}$, show that we finally have
    %
    \[
    \begin{aligned}
      \dfrac{dR}{d\tau} & = R - R^3 \\
      \dfrac{d\phi}{d\tau} & = \alpha R^2.
    \end{aligned}
    \]

  \item[e)] Given that $\tau = \epsilon^2 t = \sigma t$, show that the solution $\eta(t, \tau)$ to first order is
    %
    \[
    \eta(t, \tau) = \sqrt{\sigma} R(\tau) \exp \left( i\omega(\tau)t \right) + O(\epsilon^2)
    \]
    %
    where $\omega(\tau) = 1 + \alpha \sigma R^2(\tau)$.
    Similarly, assuming that $z_1(0) = 0$, show that the solution for $z(t, \tau)$ up to second order is given by
    %
    \[
    z(t, \tau) = \sigma R^2(\tau) \left( 1 - e^{-t} \right) + O(\epsilon^3).
    \]
    Given that $R(\tau) \to 1$ when $\tau \to +\infty$, are your results regarding the amplitude and frequency of the oscillations and the amplitude of the distortion $z(t)$ consistent with the results from the previous exercise ?
  \end{enumerate}
  
\end{problem}

\subsubsection*{Strongly nonlinear oscillators}

%% \begin{problem}
%%   Consider the FitzHugh-Nagumo model
%%   %
%%   \[
%%   \begin{aligned}
%%     \dot{v} & = v - \dfrac{v^3}{3} - w + I \\
%%     \dot{w} & = \epsilon(v + a - b w).
%%   \end{aligned}
%%   \]
%%   %
%%   This model is a simplified version of the Hodgkin-Huxley model detailing the activation and deactivation dynamics of a spiking neuron.
%%   Here, $v$ denotes the membrane potential while $w$ is known as the recovery variable.
%%   The input $I$ is the current imposed to the membrane.
%%   If $I$ is sufficiently small, no particular long-term dynamics are observed.
%%   However, when $I$ exceeds a critical value, the neuron starts to spike.
%%   These periodic spikes (when $I$ is kept constant above this threshold) can be understood as relaxation oscillations.
%%   Let us try to understand these dynamics using both theoretical and numerical analyses.

%%   \bigskip

%%   \begin{enumerate}
%%   \item[a)] Show that for $a = b = 0$, the FitzHugh-Nagumo model reduces to the classical van der Pol oscillator.
%%   \end{enumerate}
%% \end{problem}
