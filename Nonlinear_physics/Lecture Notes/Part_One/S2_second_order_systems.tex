\chapter{Second-order systems}\label{chap: second-order systems}

\section{Linear systems}\label{sec: linear systems}

\subsection{List of problems}

\subsubsection*{Linear algebra}

\begin{problem}
  Given a matrix $\bm{A} \in \mathbb{R}^{2 \times 2}$, show that its characteristic polynomial can be expressed as
  %
  \[
  \lambda^2 - \text{tr}(\bm{A}) \lambda + \det(\bm{A}) = 0
  \]
  %
  where $\text{tr}(\bm{A})$ denotes the trace of the matrix and $\det(\bm{A})$ its determinant.
\end{problem}

\bigskip

\begin{problem}
  Show that, if $\bm{A} \in \mathbb{R}^{n \times n}$ is a symmetric matrix (i.e.\ $\bm{A}^T = \bm{A}$), then all of its eigenvalues $\lambda$ are real and its eigenvectors $\bm{v}$'s form an orthonormal basis for $\mathbb{R}^n$.
\end{problem}

\bigskip

\begin{problem}
  Show that if $\bm{A}$ is a symmetric negative definite matrix (i.e.\ $\bm{A}^T = \bm{A}$ and all of its eigenvalues are negative) then
  %
  \[
  \dot{\bm{x}} = \bm{Ax}
  \]
  %
  implies
  %
  \[
  \dfrac{1}{2} \dfrac{d}{dt} \| \bm{x} \|_2^2 = - \| \bm{Rx} \|_2^2
  \]
  %
  where $\bm{R}$ is the Cholesky factor of $-\bm{A}$ (i.e.\ $\bm{A} = - \bm{R}^T \bm{R}$ where $\bm{R}^T \bm{R}$ is a symmetric positive definite matrix).
  What does the equation above implies for the evolution of the energy in the system ?
\end{problem}

\bigskip

\begin{problem}
  Show that any matrix of the form
  %
  \[
  \bm{A} = \begin{bmatrix} \lambda & a \\ 0 & \lambda \end{bmatrix}
  \]
  %
  with $a \neq 0$ has only a one-dimensional eigenspace corresponding to the eigenvalue $\lambda$.
  Then solve the system $\dot{\bm{x}} = \bm{Ax}$ and sketch the phase portrait.
\end{problem}

\bigskip

\begin{problem}
  Show that the eigenvalues of a skew-symmetric matrix
  %
  \[
  \bm{A} = \begin{bmatrix} 0 & a \\ -a & 0 \end{bmatrix}
  \]
  %
  are complex conjugate imaginary numbers.
\end{problem}

\bigskip

\begin{problem}
  Consider the linear dynamical system $\dot{\bm{x}} = \bm{Ax}$ given by
  %
  \[
  \dfrac{d}{dt} \begin{bmatrix} x \\ y \end{bmatrix}
  =
  \begin{bmatrix}
    1 & -1 \\ 1 & 1
  \end{bmatrix}
  \begin{bmatrix}
    x \\ y
  \end{bmatrix}.
  \]

  \begin{enumerate}
  \item[a)] Show that its eigenvalues are given by $\lambda = 1 \pm i$ and compute its eigenvectors $\bm{v}_1$ and $\bm{v}_2$.

  \item[b)] The general solution is $\bm{x}(t) = \alpha e^{\lambda_1 t} \bm{v}_1 + \beta e^{\lambda_2t} \bm{v}_2$.
    If $\bm{x}(0) \in \mathbb{R}^2$ then $\bm{x}(t) \in \mathbb{R}^2 \ \forall t$.
    In such a case, express $\bm{x}(t)$ purely in terms of real-valued functions.

  \item[c)] Which aspect of the dynamics do the real and imaginary parts of $\lambda$ characterize ?
  \end{enumerate}
\end{problem}

\subsubsection*{Harmonic oscillator (with and without damping)}

\begin{problem}
  Consider the canonical harmonic oscillator whose equation of motion is given by
  %
  \[
  \ddot{x} + \omega^2 x = 0.
  \]

  \begin{enumerate}
  \item[a)] Introducing a suitable potential $V(x) : \mathbb{R} \to \mathbb{R}$, show that the equation for the harmonic oscillator can be rewritten as
    %
    \[
    \ddot{x} + \dfrac{dV}{dx} = 0.
    \]

  \item[b)] Show that the equation above implies
    %
    \[
    \dfrac{d}{dt} \left( \dfrac{1}{2} \dot{x}^2 + V(x) \right) = 0.
    \]
    %
    What is the meaning of this equation ?
    What does the quantity $H(x, \dot{x}) = \dfrac{1}{2} \dot{x}^2 + V(x)$ represent from a physical point of view ?

  \item[c)] Given the expression for the potential $V(x)$ found in the previous question, show that the trajectories of the harmonic oscillator in phase space correspond to ellipses of the form $\omega^2 x^2 + y^2 = C$ with $C > 0$.
  \end{enumerate}
\end{problem}

\bigskip

\begin{problem}
  Consider the simple mechanical system shown in figure~\ref{fig: mass-spring-damper}.
  Starting from Newton's principles, the equation of motion is given by
  %
  \[
  \ddot{x} + 2k \dot{x} + \omega_0^2 x = 0
  \]
  %
  where $x$ denotes the deviation from the equilibrium position, $k > 0$ is the friction coefficient and $\omega_0$ is the natural frequency of the system in the absence of friction.
  In this exercise, we aim at classifying all of the possible dynamics exhibited by the system as we vary its parameters.
  
  \bigskip
  
  \begin{enumerate}
  \item[a)] In its current form, our model depends explicitely on two parameters, namely the friction coefficient $k$ and the natural frequency of oscillation $\omega_0$.
    Using a suitable rescaling of time, show that the model can be rewritten as
    %
    \[
    \ddot{x} + 2 \mu \dot{x} + x = 0
    \]
    %
    and give the expression of $\mu$.
    
  \item[b)] Rewrite the model above as a two-dimensional linear system of the form $\dot{\bm{x}} = \bm{Ax}$.
    
  \item[c)] Compute the characteristic polynomial of the matrix $\bm{A}$ and classify the type of fixed point for all the possible cases.
    For all cases, sketch the phase portrait of the system.
    
  \item[d)] How do these results relate to the standard notions of overdamped, critically damped and underdamped oscillations ?
  \end{enumerate}
  
\end{problem}

\begin{figure}
  \centering
  \begin{tikzpicture}[>=stealth]
      \path[pattern={Lines[angle=45,distance={8pt/sqrt(2)}]}] (-2, 5) edge ++(4,0) rectangle ++ (4, 0.5);

      \draw[decorate,decoration={coil, segment length=5pt, aspect=0.7, amplitude=4pt,
          pre=lineto, pre length=5mm, post=lineto, post length=5mm}] (0.5, 5) -- (0.5, 2.5)
      node[left, draw, minimum size=1cm, fill=blue!20] (m) {$m$};

      \draw[] (-0.5, 5) -- (-0.5, 4) {};
      \draw[] (-0.25, 4) -- (-0.75, 4) {};

      \draw[] (-0.5, 3.75) -- (-0.5, 2.5) {};
      \draw[] (-0.25, 3.75) -- (-0.75, 3.75) {};

      \draw[] (-0.25, 3.75) -- (-0.25, 4.25) {};
      \draw[] (-0.75, 3.75) -- (-0.75, 4.25) {};

      \draw[blue] (m.center-|0, 0) ++ (0, 1) -- ++ (-2, 0)
      edge[<-,edge label'=$x$,shorten >=1pt] (m.center-|-2, 0)
      (m.center-|0, 0) ++ (0, -1) -- ++ (-2, 0) 
      edge[<-,edge label=$-x$,shorten >=1pt] (m.center-|-2, 0)
      (m.east) edge[dashed] (m.east-|2, 0) 
      (m.east-|2, 0) node[right] {Equilibrium};
  \end{tikzpicture}
  \caption{Mechanical system considered in Problem 2.2}\label{fig: mass-spring-damper}
\end{figure}

\bigskip

\begin{problem}
  Consider the electric circuit shown in figure~\ref{fig: RLC circuit}.
  The Kirchoff's voltage law implies
  %
  \[
  V_R(t) + V_L(t) + V_C(t) = V_{\text{in}}(t)
  \]
  %
  where $V_R$, $V_L$ and $V_C$ are the voltages accross R, L, and C respectively and $V_{\text{in}}$ is the (possibly time-varying) voltage from the source.
  Using Ohm's law, the voltage accross the resistor can be expressed as
  %
  \[
  V_R(t) = R i(t).
  \]
  %
  Similarly, the voltage $V_L$ accross the inductor is related to the intensity of the current by
  %
  \[
  V_L(t) = L \dfrac{di}{dt}.
  \]
  %
  Finally, the intensity $i(t)$ of the current is related to the voltage accross the capacitor by
  %
  \[
  i = C \dfrac{dV_C}{dt}.
  \]
  %
  Combining all of these consitutive equations together yields the following second-order differential equation
  %
  \[
  \dfrac{d^2 V_C}{dt^2} + \dfrac{R}{L} \dfrac{d V_c}{dt} + \dfrac{V_C}{LC} = \dfrac{V_{\text{in}}}{LC}
  \]
  %
  where we'll assume that the source voltage $V_{\text{in}}$ is constant.

  \bigskip

  \begin{enumerate}
  \item[a)] Show that a natural time scale for the oscillatory behaviour of the system is given by $\tau = \sqrt{LC}$ while the damping due to the resitor occurs on a time scale $\tau = \nicefrac{R}{L}$.

  \item[b)] Show that the system admits a single fixed point given by $V_C = V_{\text{in}}$.

  \item[c)] If $V_{\text{in}} = 0$, show that we have
    %
    \[
    \dfrac{d}{dt} \left( \dfrac{1}{2} Li^2 + \dfrac{1}{2}C V_C^2 \right) = - Ri^2
    \]
    %
    where $\nicefrac{Li^2}{2}$ is the magnetic energy in the coil and $\nicefrac{CV_C^2}{2}$ is the electric energy in the condensator.
    From a physical point of view, what does the term $-Ri^2$ represent ?
  \end{enumerate}
\end{problem}

\begin{figure}
  \centering
  \begin{circuitikz}
    \draw (0, 0) to[battery1, invert] (0, 3) to [R, i>^=$i(t)$, l=$R$] (2.5, 3);
    \draw (2.5, 3) to [L, l=$L$] (2.5, 0);
    \draw (2.5, 0) to [C, l=$C$] (0, 0);
  \end{circuitikz}
  \caption{RLC circuit for Problem 2.2}\label{fig: RLC circuit}
\end{figure}

\section{Nonlinear systems}\label{sec: nonlinear systems}

\subsection{List of problems}

\subsubsection*{Conservative systems}

\begin{problem}
  Consider the Duffing oscillator whose equation is given by
  %
  \[
  \ddot{x} + \epsilon x + x^3 = 0
  \]
  %
  where $\epsilon$ can either be positive or negative.

  \bigskip

  \begin{enumerate}
  \item[a)] Show that this system is conservative and give the expression of its Hamiltonian.

  \item[b)] For $\epsilon > 0$, show that the system has a nonlinear center at its origin and sketch the phase portrait.

  \item[c)] For $\epsilon < 0$, show that the system admits three fixed points and classify them.
    Sketch the phase portrait.
  \end{enumerate}
\end{problem}

\bigskip

\begin{problem}
  Consider the \emph{Lotka-Volterra predator-prey model} given by
  %
  \[
  \begin{aligned}
    \dot{R} & = a R - b RF \\
    \dot{F} & = -c F + d RF
  \end{aligned}
  \]
  %
  where $R(t)$ denotes the population of rabbits and $F(t)$ that of foxes.
  We'll assume that all constants are positive as well as $R(t)$ and $F(t)$.

  \begin{enumerate}
  \item[a)] Discuss the biological meaning of all the terms in the model as well as its limitations.

  \item[b)] Introducing a suitable rescaling of the variables $R$ and $F$ as well as time, show that the model can be recast as
    %
    \[
    \begin{aligned}
      \dot{x} & = x - xy \\
      \dot{y} & = - \mu y + \mu xy.
    \end{aligned}
    \]

  \item[c)] Show that the system is conservative.

  \item[d)] Compute the fixed points of the system and classify them.

  \item[e)] Assuming that both $x(0)$ and $y(0)$ are non-zero, show that model predicts cycles in the populations of both species and sketch the phase portrait.

  \item[f)] Sketch the evolution of $x(t)$ and $y(t)$ as a function of time.
    Explain from a biological point of view what you observe.
  \end{enumerate}
\end{problem}

\subsubsection*{Dissipative systems}

\begin{problem}
  Consider once again the Duffing oscillator but let us now include the influence of friction.
  The equation becomes
  %
  \[
  \ddot{x} + \dot{x} - x + x^3 = 0
  \]

  \begin{enumerate}
  \item[a)] Show that the system is no longer conservative.

  \item[b)] Recast the system as two first-order equations, compute the fixed points of the system and classify them.

  \item[c)] In the vicinity of the origin, the stable manifold can be represented as $y = h(x)$.
    Assuming that $h(x) = a x + b x^2 + c x^3$, determine the coefficients of this third-order approximation.
  \end{enumerate}
\end{problem}

\subsubsection*{Modelling a pandemic}

\begin{problem}
  The recent years have been marked by the COVID-19 pandemic.
  Since the begining of the pandemic, numerous models have been proposed to forecast its evolution.
  One of the simplest, dating back to the seminal work of Kermack and McKendrick in the late 1920's and early 1930's, is known as the SIR model.
  It is given by
  %
  \[
  \begin{aligned}
    \dfrac{dS}{dt} & = -\beta \dfrac{SI}{N} \\
    \dfrac{dI}{dt} & = \beta \dfrac{SI}{N} - \gamma I \\
    \dfrac{dR}{dt} & = \gamma I
  \end{aligned}
  \]
  %
  where $N$ is the total population and $S$, $I$ and $R$ denote the number of suceptible (i.e.\ not yet infected), infected and removed (i.e.\ having recovered or died from the disease) persons in the population.
  The parameter $\beta$ characterizes the probability for susceptible individuals to become infected whenever they meet an infectious person.
  The parameter $\gamma$ models how fast an infected individual recovers (or dies) from the disease.

  \begin{enumerate}
  \item[a)] Show that $S(t) + I(t) + R(t)$ is a conserved quantity.

  \item[b)] Introducing a suitable rescaling of the variables and time, show that the model can be recast as
    %
    \[
    \begin{aligned}
      \dot{x} & = -R_0 xy \\
      \dot{y} & = R_0 xy - y \\
      \dot{z} & = y
    \end{aligned}
    \]
    %
    where $R_0$ is known as the \emph{basic reproduction number}.
    From an epidemiological point of view, what does this number represent ?

  \item[c)] The equation for $z$ being decoupled from the rest, the dynamics of $x$ and $y$ indeed forms a two-dimensional system.
    Find and classify all of the fixed points of the system.

  \item[d)] Sketch the nullclines and the phase portrait of the system.

  \item[e)] Let us assume that, as the disease first emerges, most of the population is susceptible (i.e.\ $x_0 \simeq 1$).
    An epidemic is said to occur if $y(t)$ initially increases.
    Under what condition on the basic reproduction number $R_0$ does an epidemic can occur ?

  \item[f)] At the begining of a pandemic, $x(t) \simeq 1$.
    Show that the number of infected individuals initially grows exponentially fast.

  \item[g)] If an outbreak happens, it is considered that \emph{herd immunity} has been achieved once the number of infected individuals starts to decrease.
    What fraction of the population needs to have been infected at one point or another before such a decrease happens ?

  \item[h)] At the begining of the COVID-19, the basic reproduction number was estimated to be approximately equal to 3.
    Given that France has a population of roughly 70 millions inhabitants, how many people would have had to contract the disease before herd immunity had been achieved ?
  \end{enumerate}
\end{problem}

\bigskip

\begin{problem}
  Several strategies have been proposed to mitigate the pandemic.
  Two of the most widely used strategies are social distancing and isolating infected individuals.
  A simple model for the influence of these two strategies is
  %
  \[
  \begin{aligned}
    \dot{x} & = -(R_0 - a) xy \\
    \dot{y} & = (R_0 - a) xy - b y
  \end{aligned}
  \]
  %
  where $a \geq 0$ characterizes by how much we reduced our social interactions and $b \geq 1$ how fast infected individuals are effectively removed from the population.

  \bigskip

  \begin{enumerate}
  \item[a)] Show that social distancing leads to a slower growth of the number of infected individuals at the begining of the pandemic.

  \item[b)] Rescaling time, show that isolation effectively leads to a modified basic reproduction number $R_b < R_0$.
  \end{enumerate}
\end{problem}
