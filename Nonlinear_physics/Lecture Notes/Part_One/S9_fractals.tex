\section{A primer in fractal geometry}

\subsection{List of problems}

\subsubsection*{Self-similar fractals}

\def\trianglewidth{.22\textwidth}%
\pgfdeclarelindenmayersystem{Sierpinski triangle}{
    \symbol{X}{\pgflsystemdrawforward}
    \symbol{Y}{\pgflsystemdrawforward}
    \rule{X -> X-Y+X+Y-X}
    \rule{Y -> YY}
}%

\def\kochcurve{.333\textwidth}%
\pgfdeclarelindenmayersystem{Koch curve}{
  \symbol{X}{\pgflsystemdrawforward}
  \rule{X -> X + X -- X + X}
}

\pgfdeclarelindenmayersystem{three half curve}{
  \symbol{X}{\pgflsystemdrawforward}
  \rule{X -> X + X - X - XX + X + X - X}
}

\begin{figure}
  \centering
  \subfigure[$n=0$]{
    \begin{tikzpicture}
      \draw[thick] (0,0) l-system
           [l-system={Koch curve, axiom=X, step=\kochcurve, order=0, angle=60}];
    \end{tikzpicture}    
  }
  \subfigure[$n=1$]{
    \begin{tikzpicture}
      \draw[thick] (0,0) l-system
           [l-system={Koch curve, axiom=X, step=\kochcurve/3, order=1, angle=60}];
    \end{tikzpicture}    
  }\\

  \subfigure[$n=2$]{
    \begin{tikzpicture}
      \draw[thick] (0,0) l-system
           [l-system={Koch curve, axiom=X, step=\kochcurve/9, order=2, angle=60}];
    \end{tikzpicture}    
  }
  \subfigure[$n=3$]{
    \begin{tikzpicture}
      \draw[thick] (0,0) l-system
           [l-system={Koch curve, axiom=X, step=\kochcurve/27, order=3, angle=60}];
    \end{tikzpicture}    
  }

  \caption{First four iterations for the construction of the Koch curve.}\label{fig: koch curve}
\end{figure}

\begin{problem}
  Consider the fractal object known as the Koch curve.
  The first four iterations to construct this curve are shown in figure~\ref{fig: koch curve}.
  It can be generated by the same four affine transformations again and again.

  \begin{enumerate}
  \item[a)] Find all four affine transformations needed to construct this object and show that they are contractive (i.e.\ all of their eigenvalues are inside the unit circle).

  \item[b)] Show that its perimeter goes to infinity as the number of iterations $n \to \infty$.
  \end{enumerate}
\end{problem}

\bigskip

\begin{figure}
  \centering
  \subfigure[$n=0$]{
    \begin{tikzpicture}
      \draw[thick] (0,0) l-system
           [l-system={three half curve, axiom=X, step=\kochcurve, order=0, angle=90}];
    \end{tikzpicture}    
  }
  \subfigure[$n=1$]{
    \begin{tikzpicture}
      \draw[thick] (0,0) l-system
           [l-system={three half curve, axiom=X, step=\kochcurve/4, order=1, angle=90}];
    \end{tikzpicture}    
  }\\

  \subfigure[$n=2$]{
    \begin{tikzpicture}
      \draw[thick] (0,0) l-system
           [l-system={three half curve, axiom=X, step=\kochcurve/16, order=2, angle=90}];
    \end{tikzpicture}    
  }
  \subfigure[$n=3$]{
    \begin{tikzpicture}
      \draw[thick] (0,0) l-system
           [l-system={three half curve, axiom=X, step=\kochcurve/64, order=3, angle=90}];
    \end{tikzpicture}    
  }

  \caption{First four iterations for the construction of the $\dfrac{3}{2}$ curve.}\label{fig: 3/2 curve}
\end{figure}

\begin{problem}
  Consider the curve shown in figute~\ref{fig: 3/2 curve}.
  This curve is simply known as the $\dfrac{3}{2}$ curve.
  Why is it call that way ?
\end{problem}

\bigskip

\begin{figure}
  \centering
  \subfigure[$n=0$]{
    \begin{tikzpicture}
      \fill [black] (0,0) -- ++(0:\trianglewidth) -- ++(120:\trianglewidth) -- cycle;
      \draw [draw=none] (0,0) l-system
            [l-system={Sierpinski triangle, axiom=X, step=\trianglewidth, order=0, angle=-120},fill=white];
    \end{tikzpicture}    
  }%
  \hfill
  \subfigure[$n=1$]{
    \begin{tikzpicture}
      \fill [black] (0,0) -- ++(0:\trianglewidth) -- ++(120:\trianglewidth) -- cycle;
      \draw [draw=none] (0,0) l-system
            [l-system={Sierpinski triangle, axiom=X, step=\trianglewidth/2, order=1, angle=-120},fill=white];
    \end{tikzpicture}    
  }%
  \hfill
  \subfigure[$n=2$]{
    \begin{tikzpicture}
      \fill [black] (0,0) -- ++(0:\trianglewidth) -- ++(120:\trianglewidth) -- cycle;
      \draw [draw=none] (0,0) l-system
            [l-system={Sierpinski triangle, axiom=X, step=\trianglewidth/4, order=2, angle=-120},fill=white];
    \end{tikzpicture}    
  }%
  \hfill
  \subfigure[$n=3$]{
    \begin{tikzpicture}
      \fill [black] (0,0) -- ++(0:\trianglewidth) -- ++(120:\trianglewidth) -- cycle;
      \draw [draw=none] (0,0) l-system
            [l-system={Sierpinski triangle, axiom=X, step=\trianglewidth/8, order=3, angle=-120},fill=white];
    \end{tikzpicture}    
  }

  \caption{First four iterations for the construction of the Sierpinsky triangle.}\label{fig: sierpinsky triangle}
\end{figure}

\begin{problem}
  Consider the fractal object known as the Sierpinsky triangle whose first four iterations are shown in figure~\ref{fig: sierpinsky triangle}.
  This fractal can be generated by applying repeatdly the same three affine transformations.

  \begin{enumerate}
  \item[a)] Find all three affine transformations needed to construct this figure and show that they are are contractive (i.e.\ all of their eigenvalues are inside the unit circle).

  \item[b)] Using simple geometric arguments, show that the perimeter of the Sierpinsky triangle goes to infinity as $n \to \infty$ (where $n$ is the number of iterations) while its area goes to zero.

  \item[c)] What is the Haussdorf dimension of this fractal ?
  \end{enumerate}
\end{problem}

\subsubsection*{The Mandelbrot set}

\begin{problem}[Logistic map and the Mandelbrot set]
  As discussed in \textsection~\ref{}, the logistic map
  %
  \[
  x_{k+1} = \mu x_k \left( 1 - x_k \right), \quad \text{with} \quad \mu \in \left[0, 4 \right]
  \]
  %
  is actually the Mandelbrot set generating system $z_{k+1} = z_k^2 + c$ in disguise.
  Show that it is indeed the case. (\underline{Hint} : use a simple linear transformation $z = ax + b$)
\end{problem}

\bigskip

\begin{problem}[]
  Prove by induction that any point $c \in \mathbb{C}$ such that $\vert c \vert > 2$ is not in the Mandelbrot set (i.e.\ the iteration $z_{k+1} = z_k^2 + c$ eventually diverges).
\end{problem}

\bigskip

\begin{problem}[Escape radius]
  Prove by induction that if $\vert z_k \vert > 2$ for any point $c \in \mathbb{C}$ such that $\vert c \vert < 2$, then $c$ is not in the Mandelbrot set.
\end{problem}

\bigskip

\begin{problem}[Approximating $\pi$ in a terribly inefficient way]
  $\pi$ has the tendancy of appearing in surprising ways in numerous problems where, at first, it seems like it shouldn't.
  Let us consider $c = -\dfrac{3}{4} + i \epsilon$ where $\epsilon$ is a small positive real number.
  Note that $c = -\dfrac{3}{4}$ is in the Mandelbrot set.
  It is thus expected that, as $\epsilon \to 0$, an increasing number of steps is required for the iteration to reach the escape radius $\vert z_k \vert > 2$.
  Let us denote this number by $N(\epsilon)$.
  Write down a small program that compute $N(\epsilon)$ for $\epsilon = 1, 0.1, 0.01, 0.001, \cdots$ and see what happens !

\end{problem}
