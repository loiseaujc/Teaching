%\documentclass[12pt, answers]{exam}
\documentclass[12pt]{exam}
\usepackage[utf8]{inputenc}

\usepackage[margin=1in]{geometry}
\usepackage{amsmath,amssymb}
\usepackage{multicol}
\usepackage[]{graphicx}
\usepackage[]{bm}
\usepackage[]{nicefrac}
\usepackage[]{xcolor}

\DeclareMathOperator*{\argmin}{argmin}
\DeclareMathAlphabet{\mathcal}{OMS}{cmsy}{m}{n}
\DeclareMathAlphabet\mathbfcal{OMS}{cmsy}{b}{n}
\DeclareMathOperator*{\minimize}{minimize~}
\DeclareMathOperator*{\maximize}{maximize~}
\DeclareMathOperator*{\subjectto}{subject~to~}


\newcommand{\class}{Signal Processing and Statistics}
\newcommand{\term}{Master 2}
\newcommand{\examnum}{Final Exam}
\newcommand{\examdate}{2021-2022}
\newcommand{\timelimit}{90 Minutes}

\pagestyle{head}
\firstpageheader{}{}{}
\runningheader{\class}{\examnum\ - Page \thepage\ of \numpages}{\examdate}
\runningheadrule

\usepackage[]{tikz}

\begin{document}

\noindent
\begin{tabular*}{\textwidth}{l @{\extracolsep{\fill}} r @{\extracolsep{6pt}} l}
  \textbf{\class} \\%& \textbf{Nom -- Prénom:} & \makebox[2in]{\hrulefill}\\
  % \textbf{\term} &&\\
  \textbf{\examnum} &&\\
  \textbf{\examdate} && \\
  % \textbf{Time Limit: \timelimit} & Teaching Assistant & \makebox[2in]{\hrulefill}
  \textbf{Time limit: \timelimit} %&&
\end{tabular*}\\
\rule[2ex]{\textwidth}{2pt}

This exam contains \numpages\ pages (including this cover page) and \numquestions\ exercises. \\

%% \begin{center}
%%   Barême\\
%%   \bigskip
%%   \addpoints
%%   \gradetable[v][questions]
%% \end{center}

\noindent
\rule[2ex]{\textwidth}{2pt}

\begin{questions}

  %%%%%
  %%%%%
  %%%%%     MINIMUM ENERGY CONTROL
  %%%%%
  %%%%%

  \question \textbf{Fourier series}
  \noaddpoints

  Let us consider a real-valued function $x(t)$ such that $x(t + 2\pi) = x(t)$.
  Its Fourier series representation is given by
  %
  \[
  x(t) = \dfrac{a_0}{2} + \sum_{n=1}^{\infty} \left( a_n \cos(2\pi n f_0 t) + b_n \sin(2\pi n f_0 t) \right).
  \]
  %
  The coefficients $a_0$, $a_n$ and $b_n$ are given by
  %
  \[
  \begin{aligned}
    a_0 & = \dfrac{1}{\pi} \int_{-\pi}^\pi x(t) \ \mathrm{d}t, \\
    a_n & = \dfrac{1}{\pi} \int_{-\pi}^\pi x(t) \cos(2\pi n f_0 t) \ \mathrm{d}t, \\
    b_n & = \dfrac{1}{\pi} \int_{-\pi}^\pi x(t) \sin(2\pi n f_0 t) \ \mathrm{d}t,
  \end{aligned}
  \]
  %
  with $f_0 = \nicefrac{1}{T}$ its fundamental frequency.

  \begin{parts}
    \part[2] Prove that the basis of real harmonic oscillations
    %
    \[
    \sin(2\pi n f_0 t), \quad \cos(2\pi n f_0 t), \quad n = 1, 2, \cdots
    \]
    %
    forms an orthogonal basis, i.e. their inner product is equal to $0$ if $m \neq n$ and non-zero otherwise.

    \begin{solution}
      {\color{blue}
        Solution
      }
    \end{solution}

    \part[1] Using the results of the previous question, find formulas for the amplitudes $c_n$ and phases $\theta_n$ in the expansion of the periodic signal $x(t)$ in terms of only cosines, i.e.
    %
    \[
    x(t) = \sum_{n=0}^\infty c_m \cos(2\pi n f_0 t + \theta_n).
    \]

    \part[2] Using these results, show that
    %
    \[
    \dfrac{a_0^2}{2} + \sum_{n=1}^{\infty} \left( a_n^2 + b_n^2 \right) = \dfrac{1}{\pi} \int_{-\pi}^{\pi} x^2(t) \ \mathrm{d}t.
    \]
    %
    This result is known as \emph{Parseval's theorem}.
  \end{parts}

  \question \textbf{Stochastic systems}
  \noaddpoints

  Consider the following continuous-time linear system
  %
  \[
  \dot{x} = -\alpha x + w
  \]
  %
  with $x \in \mathbb{R}$ the state variable, $w \in \mathbb{R}$ the noise, $\alpha > 0$ and $\sigma > 0$.

  \begin{parts}
    \part[1] Let us assume the autocorrelation function of the noise $w$ is given by
    %
    \[
    R_{ww}(\tau) = \sigma^2 \delta(\tau).
    \]
    %
    What does this tell you about the properties of the noise process $w$ ?

    \part[2] The impulse response of $\dot{x} = \alpha x$ is given by
    %
    \[
    h(\tau) = \begin{cases}
      e^{-\alpha \tau} \quad \text{if } \tau \geq 0 \\
      0 \quad \text{otherwise}.
      \end{cases}
    \]
    %
    The cross-correlation function between $w(t)$ and $x(t)$ is given by
    %
    \[
    \begin{aligned}
      R_{wx}(\tau) & = h(\tau) * R_{ww}(\tau) \\
      & = \int_{-\infty}^{\infty} h(t) R_{ww}(\tau-t) \ \mathrm{d}t.
    \end{aligned}
    \]
    %
    Give the analytical expression of $R_{wx}(\tau)$.

    \part[2] The auto-correlation of $x(t)$ is given by
    %
    \[
    R_{xx}(\tau) = h(-\tau) * h(\tau) * R_{ww}(\tau),
    \]
    %
    Given the analytical expression of $R_{xx}(\tau)$.
    What is the typical time-scale over which the signal $x(t)$ is correlated ?
  \end{parts}


\question \textbf{Bivariate statistics}
\noaddpoints	

The value of 2 statistical variables $X$ and $Y$ is given in table \ref{table:data_exercise_2} for 5 people.

  \begin{table}[h]
	\centering
	\begin{tabular}{|l|c|c|}
		\hline
		%& Velocity   & Turbulent kinetic energy & Reynolds shear stress& Skin friction coefficient \\
		& X    & Y  \\\hline
		Inividual 1  &  3 & 12  \\\hline 
		Inividual 2  &  4 & 14  \\\hline
		Inividual 3  &  2 & 8   \\\hline
		Inividual 4  &  5 & 19   \\\hline
		Inividual 5  &  3 & 11  \\\hline
	\end{tabular}
	\caption{Statistical variables $X$ and $Y$ evaluated for 5 people.}
	\label{table:data_exercise_2}
    \end{table}

\begin{parts}
	
	\part[1] Compute the marginal arithmetic means $\overline{X}$ and $\overline{Y}$ for each variable.
	
	\begin{solution}
	\end{solution}

	\part[1] Compute the marginal standard deviations $\sigma(X)$ and $\sigma(Y)$ for each variable.

    \begin{solution}
    \end{solution}

	\part[1] Compute the covariance $\text{cov}(X,Y)$ between $X$ and $Y$ .

    \begin{solution}
    \end{solution}

	\part[1] Suppose that a linear correlation holds between X and Y. Compute the equation of the regression line, X being the explanatory variable.

    \begin{solution}
    \end{solution}

	\part[1] Compute the correlation coefficient between $X$ and $Y$. What do you conclude?

    \begin{solution}
    \end{solution}
	
\end{parts}  	

  %%%%%
  %%%%%
  %%%%%     PRINCIPAL COMPONENT ANALYSIS
  %%%%%
  %%%%%

  \question \textbf{Principal component analysis}
  \noaddpoints
  
  Consider the table \ref{table:data_exercise_1}, in which the size and weight of 6 people is given. In this exercise, we apply the Principal Component Analysis (PCA) on this set of data.
  
  \begin{table}[h]
  	\centering
  	\begin{tabular}{|l|c|c|}
  		\hline
  		%& Velocity   & Turbulent kinetic energy & Reynolds shear stress& Skin friction coefficient \\
  		&size [cm]    & weight [kg]  \\\hline
  		Inividual 1  & 51 & 162  \\\hline 
  		Inividual 2  & 64  & 165  
  		\\\hline
  		Inividual 3  &  60  & 150   \\\hline
  		Inividual 4  &  90  & 190   \\\hline
  		Inividual 5  &  95  & 180   \\\hline
  		Inividual 6  &  85  & 185   \\\hline
  	\end{tabular}
  	\caption{Size and weight of 6 people.}
  	\label{table:data_exercise_1}
  \end{table}

We note $n$ the number of individuals and $p$ the number of variables. We note $X \in M_{n,p}(\mathbb{R})$, the data matrix gathering the data of table \ref{table:data_exercise_1} (each line of $X$ corresponds to a specific individual, the first column of $X$ corresponds to the variable "size" and the second one corresponds to the variable "weight"). 

\begin{parts}
	\part[1] Compute the line vectors $\overline{X}^{T}$, $\text{Var}(X)^{T}$ and $\sigma(X)^T$ gathering respectively the marginal arithmetic means, variances and standard deviations of each variable. 
	
	\begin{solution}
    \end{solution}

	\part[1] We note m the number of principal components that can be computed. What is the value of m? How many principal components must be considered to describe 100\% of the variability of the data. 
    \begin{solution}
    \end{solution}

	\part[1] Compute the centered data matrix $X_c$ and the covariance matrix of the centered data $\text{Cov}(X_c)$.
    \begin{solution}
    \end{solution}

	\part[1] We note $q_i$ ($1 \leq i \leq m$) the m principal components. Compute the m principal components. For each value of i, compute the part (in \%) of the variability of the data explained by the first i principal components.
    \begin{solution}
    \end{solution}

	\part[1] Express the centered data matrix $Y$ of the data coordinates in the principal components coordinate system.
    \begin{solution}
    \end{solution}

\end{parts}  	

\end{questions}

\end{document}
