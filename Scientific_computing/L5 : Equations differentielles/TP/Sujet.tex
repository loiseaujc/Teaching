 
\documentclass[a4paper,12pt]{article} % This defines the style of your paper

\usepackage[top = 2.5cm, bottom = 2.5cm, left = 2.5cm, right = 2.5cm]{geometry}
\usepackage[T1]{fontenc}
\usepackage[utf8]{inputenc}
\usepackage{multirow} % Multirow is for tables with multiple rows within one cell.
\usepackage{booktabs} % For even nicer tables.
\usepackage{graphicx} 
\usepackage{setspace}
\setlength{\parindent}{0in}
\usepackage{float}
\usepackage{fancyhdr}

\usepackage{tikz}
\usepackage{pgfplots}
\usepgfplotslibrary{polar}

\usepackage{hyperref,graphicx,lmodern}
\usepackage{xcolor}
\usepackage{nicefrac}
\usepackage{upgreek}
\usepackage[]{bm}
\usepackage{amsmath}
\usepackage[]{mathtools}

\graphicspath{{imgs/}}

\usepackage[]{listings}

\definecolor{codegreen}{rgb}{0,0.6,0}
\definecolor{codegray}{rgb}{0.5,0.5,0.5}
\definecolor{codepurple}{rgb}{0.58,0,0.82}
\definecolor{backcolour}{rgb}{1.0, 1.0, 1.0}

\lstdefinestyle{mystyle}{
  backgroundcolor=\color{backcolour},
  commentstyle=\color{codegreen},
  keywordstyle=\color{magenta},
  numberstyle=\tiny\color{codegray},
  stringstyle=\color{codepurple},
  basicstyle=\ttfamily\footnotesize,
  breakatwhitespace=false,
  breaklines=true,
  captionpos=b,
  keepspaces=true,
  numbers=left,
  numbersep=5pt,
  showspaces=false,
  showstringspaces=false,
  showtabs=false,
  tabsize=2
}

\lstset{style=mystyle}


%%%%%%%%%%%%%%%%%%%%%%%%%%%%%%%%%%%%%%%%%%%%%%%%
% 3. Header (and Footer)
%%%%%%%%%%%%%%%%%%%%%%%%%%%%%%%%%%%%%%%%%%%%%%%%
\pagestyle{fancy} % With this command we can customize the header style.
\fancyhf{} % This makes sure we do not have other information in our header or footer.

\lhead{\footnotesize QM: Homework 1}% \lhead puts text in the top left corner. \footnotesize sets our font to a smaller size.

\rhead{\footnotesize Lastname 1, Lastname 2 (\& Lastname 3)} %<---- Fill in your lastnames.

\cfoot{\footnotesize \thepage}

\begin{document}


%%%%%%%%%%%%%%%%%%%%%%%%%%%%%%%%%%%%%%%%%%%%%%%%
%%%%%%%%%%%%%%%%%%%%%%%%%%%%%%%%%%%%%%%%%%%%%%%%

%%%%%%%%%%%%%%%%%%%%%%%%%%%%%%%%%%%%%%%%%%%%%%%%
% Title section of the document
%%%%%%%%%%%%%%%%%%%%%%%%%%%%%%%%%%%%%%%%%%%%%%%%

% For the title section we want to reproduce the title section of the Problem Set and add your names.

\thispagestyle{empty} % This command disables the header on the first page. 

\begin{tabular}{p{15.5cm}} % This is a simple tabular environment to align your text nicely 
{\large \bf Introduction au calcul scientifique} \\
Arts \& Métiers Sciences \& Technologies \\
Année universitaire : 2020-2021 \\
\hline % \hline produces horizontal lines.
\\
\end{tabular} % Our tabular environment ends here.

\vspace*{0.3cm} % Now we want to add some vertical space in between the line and our title.

\begin{center} % Everything within the center environment is centered.
	{\Large \bf Equations différentielles ordinaires} % <---- Don't forget to put in the right number
	\vspace{2mm} % <---- Fill in your names here!
\end{center}  

\vspace{0.4cm}

%%%%%%%%%%%%%%%%%%%%%%%%%%%%%%%%%%%%%%%%%%%%%%%%
%%%%%%%%%%%%%%%%%%%%%%%%%%%%%%%%%%%%%%%%%%%%%%%%

\subsection*{Exercice 1 : Simuler un tir de canon}

Considérons à nouveau le problème d'artillerie utilisé lors de la dernière séance de TP.
La trajectoire d'un projectile tiré par un canon d'artillerie peut être déterminée à partir des équations du mouvement
%
\[
\begin{aligned}
  \ddot{x} & = -\alpha f(\dot{x}, \dot{y})\dot{x}\\
  \ddot{y} & = -\alpha f(\dot{x}, \dot{y}) \dot{y} - 1
\end{aligned}
\]
%
où $f(\dot{x}, \dot{y})$ modélise les frottements.
On adjoint également les conditions initiales
%
\[
x(0) = y(0) = 0, \quad \dot{x}(0) = \cos(\theta) \quad \text{et} \quad \dot{y}(0) = \sin(\theta)
\]
%
où $\theta$ est l'angle de hausse du canon.
Nous avons vu au TP précédent qu'en absence de frottement (i.e.\ $\alpha = 0$), la trajectoire était donnée par
%
\[
y(x, \theta) = -\dfrac{1}{2} \dfrac{x^2}{\cos(\theta)} + \tan(\theta)x
\]
%
Pour des frottements linéaires (i.e.\ $f(\dot{x}, \dot{y}) = 1$), la trajectoire est maintenant donnée par
%
\[
y(x, \theta, \alpha) = \left( \tan(\theta) + \dfrac{1}{\alpha \cos(\theta)} \right) x + \dfrac{1}{\alpha^2} \mathrm{ln} \left( 1 - \dfrac{\alpha}{\cos(\theta)}x \right).
\]
%
L'objectif de cet exercice est alors de simuler le système et de comparer les prédictions de notre simulation numérique avec ces solutions analytiques.

\subsubsection*{Absence de frottement}

En absence de frottement, les équations du mouvement se réduisent à
%
\[
\begin{aligned}
  \ddot{x} & = 0 \\
  \ddot{y} & = -1
\end{aligned}
\]
%
auxquelles on adjoint les conditions initiales décrites précédemment.
Il s'agit d'un système de deux équations du second ordre.
Afin de le simuler à l'aide de \verb+scipy+, il est nécessaire tout d'abord de le transformer en un système d'équations du premier ordre.
Pour cela, introduisons les variables suivantes :
%
\[
x_1 = x, \quad x_2 = \dot{x}, \quad x_3 = y \quad \text{et} \quad x_4 = \dot{y}.
\]

\begin{enumerate}
\item Montrez que le système de deux équations du second ordre est équivalent au système suivant
  %
  \[
  \begin{aligned}
    \dot{x}_1 & = x_2 \\
    \dot{x}_2 & = 0 \\
    \dot{x}_3 & = x_4 \\
    \dot{x}_4 & = -1
  \end{aligned}
  \]
  %
  avec les conditions initiales données par
  %
  \[
  x_1 = x_3 = 0, \quad x_2 = \cos(\theta) \quad \text{et} \quad x_4 = \sin(\theta).
  \]

\item Ecrivez la fonction \verb+python+ correspondant à ce système dynamique.
  L'entête de la fonction doit être la suivante
  %
  \begin{lstlisting}[language=Python]
    def sans_frottement(t, u):
  \end{lstlisting}

\item A l'aide de la fonction \verb+solve_ivp+ du package \verb+scipy.integrate+, écrivez un script \verb+python+ permettant de simuler le système.
  On prendra les paramètres suivant :
  %
  \begin{itemize}
  \item Angle de hausse $\theta = \nicefrac{\pi}{4}$,
  \item Temps d'intégration : \verb+tspan = (0.0, 10.0)+
  \item Valeurs de $t$ pour lesquelles on souhaite avoir la trajectoire : \\\verb+t_eval = np.linspace(tspan[0], tspan[1], 128)+
  \end{itemize}

\item En supposant que vous avez appelée \verb+sol+ la variable de retour de \verb+solve_ivp+, vous pouvez accéder à la solution à l'aide de \verb+sol.y+.
  En utilisant \verb+matplotlib.pyplot+, tracez la trajectoire du projectile (i.e.\ $x_3(t)$ en fonction de $x_1(t)$) et comparez avec la solution analytique.
  Vous penserez à ajouter des titres aux axes ainsi qu'une légende.
\end{enumerate}

\subsubsection*{Frottement linéaire}

Intéressons-nous maintenant au cas où les frottements sont linéaires.
Physiquement, cela correspond à un projectile se déplaçant à basse vitesse.
Les équations du mouvement sont données par
%
\[
\begin{aligned}
  \ddot{x} & = -\alpha \dot{x} \\
  \ddot{y} & = -\alpha \dot{y} - 1
\end{aligned}
\]
%
avec les conditions initiales mentionnées au début de l'exercice.

\begin{enumerate}
\item En introduisant les mêmes variables que précédemment, montrez que le système équivalent d'équations du premier ordre est donné par
  %
  \[
  \begin{aligned}
    \dot{x}_1 & = x_2 \\
    \dot{x}_2 & = -\alpha x_2 \\
    \dot{x}_3 & = x_4 \\
    \dot{x}_4 & = -\alpha x_4 - 1.
  \end{aligned}
  \]

\item Ecrivez la fonction \verb+python+ correspondant à ce nouveau système.
  L'entête de la fonction doit être la suivante
  %
  \begin{lstlisting}[language=Python]
    def frottement_lineaire(t, u, alpha):
  \end{lstlisting}

\item Simulez le système dans les mêmes conditions qu'à l'étape précédente.
  On choisira par ailleurs $\alpha = 0.01$.

\item En supposant que vous avez appelée \verb+sol+ la variable de retour de \verb+solve_ivp+, vous pouvez accéder à la solution à l'aide de \verb+sol.y+.
  En utilisant \verb+matplotlib.pyplot+, tracez la trajectoire du projectile (i.e.\ $x_3(t)$ en fonction de $x_1(t)$) et comparez avec la solution analytique.
  Vous pouvez également ajouter la solution du cas sans frottement afin de les comparer.
  Vous penserez à ajouter des titres aux axes ainsi qu'une légende.
\end{enumerate}


%%%%%
%%%%%
%%%%%
%%%%%
%%%%%

\subsection*{Orbite planétaire}

La force gravitationnelle exercée par un corps de masse $M$ et un corps de masse $m$ est de la forme
%
\[
\bm{F} = -\dfrac{GMm}{r^3} \bm{r},
\]
%
où $G$ est la constante de gravitation.
En supposant que la masse $M$ est immobile et à l'origine de notre système de coordonnées, le mouvement du corps de masse $m$ se mouvant sous l'action de cette unique force obéit alors à l'équation du mouvement
%
\[
\ddot{\bm{r}} = -\dfrac{GMm}{r^3} \bm{r}
\]
%
où $\bm{r}$ est le vecteur position de la masse $m$ et $r = \sqrt{x^2 + y^2}$ sa distance à l'origine.
Il est possible encore une fois grâce à un changement de variables de ré-écrire ce système comme
%
\[
\ddot{\bm{r}} = -\dfrac{1}{r^3} \bm{r}.
\]

\end{document}
